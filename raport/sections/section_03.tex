\section{Wstęp}
\noindent
Celem systemu jest zdalny pomiar temperatury i wilgotności oraz ich wizualizacja i nadzór progowy. \\
\noindent
Komunikacja z urządzeniem realizowana jest w sieci LoRaWAN z wykorzystaniem TTN, a dane trafiają do bazy InfluxDB i są prezentowane w Grafanie. \\
\noindent
Zaprojektowano ścieżkę zwrotną (downlink) do sygnalizacji alarmu na urządzeniu końcowym. \\
\noindent
Całość uruchamiana jest lokalnie w kontenerach, z konfiguracją przekazywaną przez zmienne środowiskowe.

\subsection{Zakres i cele}
\noindent
Zakres obejmuje firmware węzła LoRaWAN, odbiór uplinku po MQTT, zapis do bazy czasowej oraz wizualizację i alarmowanie. \\
\noindent
System zapewnia jednolitą ścieżkę danych dla wielu pomiarów i umożliwia reakcję na przekroczenia progów.

\subsection{Zastosowane protokoły}
\noindent
LoRaWAN. \\
\noindent
MQTT (TTN). \\
\noindent
HTTP/HTTPS (API InfluxDB oraz webhook Grafany). \\
\noindent
TLS (szyfrowanie połączeń MQTT).
