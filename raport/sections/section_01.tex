\section{Implementacja i konfiguracja}
\noindent
Moduł \texttt{main.py} inicjalizuje klienta MQTT, uruchamia pętlę subskrypcji uplinku oraz równolegle serwer Flask dla webhooków. \\
\noindent
Moduł \texttt{uplink.py} dekoduje payload, wylicza temperaturę i wilgotność oraz zapisuje pomiar w linii \texttt{measurements\_ttn,device=... T=...,H=...}. \\
\noindent
Moduł \texttt{webhook.py} reaguje na alerty Grafany i steruje downlinkiem, a \texttt{downlink.py} publikuje komendy LED do TTN. \\
\noindent
Środowisko uruchomieniowe oparto o \texttt{docker-compose} z usługami InfluxDB i Grafana oraz konfiguracją przez zmienne środowiskowe.

\subsection{Inicjalizacja klienta MQTT}
\noindent
Konfiguracja połączenia z TTN wykonywana jest na starcie aplikacji.
\begin{minted}{python}
mqtt_client = mqtt.Client(callback_api_version=CallbackAPIVersion.VERSION2)
mqtt_client.username_pw_set(USERNAME, KEY)
mqtt_client.tls_set()
mqtt_client.on_message = on_message
mqtt_client.connect(HOST, PORT, keepalive=60)
\end{minted}
\noindent
Wątek serwera Flask uruchamiany jest niezależnie od pętli MQTT.

\begin{minted}{python}
flask_thread = threading.Thread(target=run_flask)
flask_thread.start()
initialize_mqtt_client()
\end{minted}

\subsection{Subskrypcja uplinku}
\noindent
Subskrypcja ogranicza się do urządzenia o wskazanym \texttt{DEV\_EUI}.
\begin{minted}{python}
sub = f"v3/{APP_ID}@ttn/devices/{DEV_EUI}/up"
mqtt_client.subscribe(sub)
\end{minted}

\subsection{Publikacja downlink}
\noindent
Komenda LED kodowana jest do base64 i wysyłana na temat TTN.
\begin{minted}{python}
payload_bytes = bytes([command])
payload_b64 = base64.b64encode(payload_bytes).decode("ascii")
downlink_msg = {
		"downlinks": [
				{
						"f_port": 1,
						"frm_payload": payload_b64,
						"priority": "NORMAL"
				}
		]
}
client.publish(topic, json.dumps(downlink_msg))
\end{minted}

\subsection{Usługi w kontenerach}
\noindent
InfluxDB i Grafana uruchamiane są jako osobne usługi, a konfiguracja jest przekazywana z pliku środowiskowego.
\begin{minted}{yaml}
services:
	influxdb:
		image: influxdb:2.7ch
	grafana:
		image: grafana/grafana:latest
\end{minted}
\noindent
Istotne zmienne środowiskowe są wymagane do poprawnej konfiguracji usług i dostępu do TTN.
\begin{minted}{text}
INFLUX_HOST              		- adres URL InfluxDB
INFLUXDB_INIT_ORG        		- organizacja w InfluxDB
INFLUXDB_INIT_BUCKET     		- bucket dla pomiarów
INFLUXDB_INIT_ADMIN_TOKEN		- token do autoryzacji zapisu
MQTT_HOST                		- adres brokera TTN
MQTT_PORT                		- port połączenia MQTT
MQTT_USERNAME            		- nazwa użytkownika TTN
MQTT_KEY                 		- klucz API TTN
MQTT_APP_ID              		- identyfikator aplikacji TTN
MQTT_DEV_EUI             		- identyfikator urządzenia
\end{minted}

\subsection{Struktura modułów}
\noindent
Podział na moduły upraszcza utrzymanie i wydziela odpowiedzialności.
\begin{minted}{text}
main.py      - start aplikacji, MQTT, Flask
uplink.py    - odbiór i zapis do InfluxDB
downlink.py  - budowanie i wysyłka downlink
webhook.py   - obsługa alertów Grafany
main.ino     - firmware węzła LoRaWAN
\end{minted}
